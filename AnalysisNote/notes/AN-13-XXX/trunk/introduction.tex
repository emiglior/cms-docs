%%%%%%%%%%%%%%%%%%%%%%%%
\section{Introduction}
%%%%%%%%%%%%%%%%%%%%%%%%
\FIXME{Outline of Introduction}
\begin{itemize}
\item bla-bla on uncertainties in track reconstruction
\item definition of the datasets used
\item event/track selection
\item outline of the structure of the note
\end{itemize}
%The measurement of track momentum is affected by the reconstruction \ldots
%capability and our limited knowledge about the physical configuration
%of the detector. The main sources of biases in the momentum measurement are from residual misalignments and
%weak-modes, imprecisions in the magnetic field model, mismodelings of the material
%distribution/density. In this note the MuScleFit algorithm~\cite{XXX} is used to correct thise effects and extract an estimate %of the transverse momentum resolution. The method is based on an unbinned maximum
%likelihood fit to calibrate the data with a reference model of the J/y.
%This note is organized as follows: first a 
