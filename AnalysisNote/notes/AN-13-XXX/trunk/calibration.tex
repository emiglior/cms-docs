%%%%%%%%%%%%%%%%%%%%%%%%%%%%%
\section{Calibration strategy}
%%%%%%%%%%%%%%%%%%%%%%%%%%%%
The MuScleFit algorithm~\cite{CMS_AN_2010-059} uses the spectrum of dimuons from the decay of
reference resonances to determine biases in the momentum assigned to
charged tracks and to estimate its resolution.
For the calibration procedure described in this note, dimuons from the decay of
Z bosons were used. 
The invariant mass spectrum of the dimuons 
was modeled as the sum of a signal obtained from a NNLO
calculation~\cite{Dittmaier:2009cr}, adapted to kinematic cuts close to
those used in the selection of reconstructed muons, and an
exponentially falling background. The decay constants of the
exponential were determined separetely in 9 bins depending on the
pseudorapidities of the two muons.
\FIXME expand description of the background model 
% $\eta(\mu^+)$ $\eta(\mu^-)$ 
%\FIXME description of MF strategy (scale no $p_0$, resolution, scale $p_0$ only, resolution) /ingredients\\
%\FIXME quote SDittmaier
%The ansatz functions describing the scale
%corrections and the resolution entering in the construction of
%the likelihood are detailed below.
%%%%%%%%%%%%%%%%%%%%%%%%%%%%%%%%%%%%%%%%%%%%%%%%%%%%%%
\subsection{Ansatz functions for scale corrections and resolution}
%%%%%%%%%%%%%%%%%%%%%%%%%%%%%%%%%%%%%%%%%%%%%%%%%%%%%%
The bias in the assignment of the momentum is mainly
related to geometrical effects, e.g. deformations of the tracker
geometry used in the reconstruction nevertheless present after the alignment procedure.
For this reason scale corrections were implemented as corrections to the curvature
$\kappa=q/p_T$ of the track.
We modeled these corrections with an ansatz function defined in five
bins of the muon pseudorapidity: 
\begin{equation}
\kappa' = (1+p_0) \left( \kappa -\frac{\delta}{2} - C_j(\eta,\varphi)  \right),
\label{eq:scale_function}
\end{equation}
where $j$ is an index running on the $\eta$ bin: [-2.4,-2.1],
[-2.1,-1.5], [-1.5,+1.5], [+1.5,+2.1] and [+2.1,+2.4].\\
The terms in Eq.~\ref{eq:scale_function} are:
\begin{itemize}
\item $p_0$ corresponding to a global correction to the scale accounting for effects
  like inaccurate knowledge of the magnetic field;
\item $\delta$ representing an absolute bias in the curvature, e.g. a
  bias on the transverse momentum of the track different for negative
  and positive muons;
\item $C_j(\eta,\varphi)$ accounting for residual misalignment effects
  in each of the five $\eta$ bins.
  The functional form 
  \[
  C_j(\eta,\varphi) = a_{1,j} \sin(\varphi+\varphi_{1,j}) + a_{2,j}
  \sin(2\varphi+\varphi_{2,j}) + b_j(\eta-\eta_{0,j}) + b_{0,j} 
  \]
  was choosen to model the weak modes more frequently
  found in the post-alignment geometry, namely the sagitta (described
  by $a_{1,j}$), the
  twist (described by $b_j$) and the elliptical (described
  by $a_{2,j}$) deformations~\footnote{
    The three weak modes correspond to the following parametric deformations:
    $r\Delta\varphi = c_s \cos\varphi$ {\sl sagitta},
    $\Delta\varphi = c_t z$  {\sl twist} and 
    $\Delta r = r (1- c_e \cos 2\varphi) $  {\sl elliptical}
    with $c_s$, $c_t$ and $c_e$ being the approriate constants.}. 
  The elliptical deformations were considered null everywhere apart from 
  the first and last $\eta$ bin.
\end{itemize}
Table~\ref{tab:scale_parameters} shows the values of the fitted
parameters found by MuScleFit. 
Coefficients $a_{1,j}$, $a_{2,j}$ and $b_j$ with magnitude smaller than
0.000001 GeV$^{-1}$ were considered null. Similarly values
of $p_0$ smaller than 0.0050 were considered zero as they 
are consistent, within the systematic uncertainty 
computed in Section~\ref{sec:systematics}, with the null value.
%%%%%%%
\begin{sidewaystable}[hbH]
\begin{center}
\caption{Values of the fitted parameters for the scale
  correction~(Eq.~\ref{eq:scale_function}) for data and simulation
  samples at 7 TeV and 8 TeV. The five sections in the lower part
  of the table correspond to the five $\eta$ bins [-2.4,-2.1],
[-2.1,-1.5], [-1.5,+1.5], [+1.5,+2.1] and [+2.1,+2.4]. The values of
the $b_0$ parameters are fixed to guarantee the continuity of the
correction function at the boundaries between the $\eta$ bins.\label{tab:scale_parameters}} 
\begin{tabular}{|l|c|c|c|c|c|}
\hline
Sample & 2011\_DATA\_44X & 2011\_MC\_44X & 2012ABC\_DATA\_ReReco\_53X & 2012D\_DATA\_ReReco\_53X & 2012\_MC\_53X\_smearReReco \\
\hline
$p_0$ & -0.00122  & 0.00000  & -0.00139  & -0.00135  & 0.00000  \\
\hline
$\delta$ & 0.00004  & 0.00004  & 0.00004  & 0.00004  & 0.00005  \\
\hline
$a_1$ & 0.00080  & 0.00022  & 0.00028  & 0.00025  & 0.00027  \\
$\phi_1$ & 1.33254  & 0.28482  & 1.21602  & 1.21894  & 0.14179  \\
$a_2$ & 0.00041  & 0.00030  & 0.00019  & 0.00017  & 0.00023  \\
$\phi_2$ & 1.79848  & -1.68476  & 1.78374  & 1.93410  & -1.71046  \\
$b$ & -0.00013  & -0.00007  & -0.00004  & -0.00005  & 0.00000  \\
%$\eta_0$ (fixed) & -2.10000  & -2.10000  & -2.10000  & -2.10000  & -2.10000  \\
$b_0$ & 0.00010  & -0.00001  & 0.00003  & 0.00004  & -0.00002  \\
\hline
$a_1$ & 0.00009  & 0.00006  & 0.00002  & 0.00000  & 0.00001  \\
$\phi_1$ & 1.17708  & -1.13170  & 0.26269  & 0.15297  & -1.04015  \\
$b$ & -0.00024  & 0.00003  & -0.00008  & -0.00009  & 0.00004  \\
%$\eta_0$ (fixed) & -1.50000  & -1.50000  & -1.50000  & -1.50000  & -1.50000  \\
$b_0$ & -0.00004  & 0.00000  & -0.00002  & -0.00002  & 0.00000  \\
\hline
$a_1$ & 0.00015  & 0.00007  & 0.00007  & 0.00007  & 0.00007  \\
$\phi_1$ & -1.30574  & -1.75023  & -1.24722  & -1.39464  & -1.64733  \\
$b$ & 0.00003  & 0.00000  & 0.00001  & 0.00001  & 0.00000  \\
\hline
$a_1$ & 0.00001  & 0.00013  & 0.00002  & 0.00001  & 0.00003  \\
$\phi_1$ & 0.89885  & -1.40495  & 0.21788  & 1.17093  & -1.68410  \\
$b$ & -0.00018  & 0.00000  & -0.00003  & -0.00003  & 0.00000  \\
%$\eta_0$ (fixed) & 1.50000  & 1.50000  & 1.50000  & 1.50000  & 1.50000  \\
$b_0$ & 0.00004  & 0.00000  & 0.00002  & 0.00002  & 0.00000  \\
\hline
$a_1$ & 0.00058  & 0.00014  & 0.00021  & 0.00032  & 0.00018  \\
$\phi_1$ & 1.85334  & -1.42615  & 1.94164  & 1.89407  & -0.94791  \\
$a_2$ & 0.00028  & 0.00001  & 0.00012  & 0.00008  & 0.00012  \\
$\phi_2$ & -0.84138  & 0.78290  & -1.10183  & -1.23759  & 0.38600  \\
$b$ & -0.00020  & -0.00001  & -0.00008  & -0.00010  & 0.00000  \\
%$\eta_0$ (fixed) & 2.10000  & 2.10000  & 2.10000  & 2.10000  & 2.10000  \\
$b_0$ & -0.00006  & 0.00000  & -0.00001  & 0.00000  & 0.00000  \\
\hline
\hline
\end{tabular}
\end{center}
\end{sidewaystable}

The resolution on $p_T$ was modeled as the sum in quadrature of two terms
\begin{equation}
\frac{\sigma(p_T)}{p_T}=q_0 p_T \oplus q_j
\label{eq:resolution_function}
\end{equation}
where the parameters $q_j$, representing multiple Coulomb scattering
effects, were computed separately in 12 equally spaced bins in $\eta$. 
%
\subsection{Fit strategy}
% 
The maximization of the likelihood was performed in two steps. 
First, all the parameters of the correction
function, with the exception of the global scale $p_0$, were estimated 
for realistic values of the resolution function. In the second step
the parameters of the resolution function were determined together
with the global scale $p_0$. 
The fitted parameters were found to be stable against further
iterations of the maximization procedure.


